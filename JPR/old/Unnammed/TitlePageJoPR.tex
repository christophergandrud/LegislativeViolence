%%%%%%%%%%%%%%%%%%%%%%%%%%%%
% Two Sword Lengths Apart
% 10 December 2014
%%%%%%%%%%%%%%%%%%%%%%%%%%%%

% !Rnw weave = knitr

\documentclass[a4paper]{article}
\usepackage{fullpage}
\usepackage{lscape}
\usepackage[authoryear]{natbib}
\usepackage{setspace}
    \doublespacing
\usepackage{hyperref}
\hypersetup{
    colorlinks,
    citecolor=black,
    filecolor=black,
    linkcolor=cyan,
    urlcolor=cyan
}
\usepackage{booktabs}
\usepackage{dcolumn}
\usepackage{url}
\usepackage{tikz}
\usepackage{todonotes}
\usepackage{verbatim}
\usepackage{endnotes}
\usepackage{graphicx}

\setlength{\belowcaptionskip}{0.5cm}

%%%%%%% Title Page %%%%%%%%%%%%%%%%%%%%%%%%%%%%%%%%%%%%%%%%%%%%
\title{Two Sword Lengths Apart: Credible Commitment Problems and Physical Violence in Multi-party Elected National Legislatures}

\author{Christopher Gandrud \\
                {\emph{Hertie School of Governance}}\footnote{Post-doctoral Researcher. Friedrichstra{\ss}er 180. 10117 Berlin, Germany. Email: \href{mailto:gandrud@hertie-school.org}{gandrud@hertie-school.org}. Thank you to Gary Cox, Simon Hix, Shirin Rai, Carole Spary, as well as seminar participants at the Hertie School of Governance and Yonsei University for very helpful comments and insights. I would also like to thank Hortense Badarani for excellent research assistance and my students at the LSE for inspiration. Full replication files--including data and source code--can be found at \url{https://github.com/christophergandrud/LegislativeViolence}. An earlier version was circulated under the title ``Two Sword Lengths: Losers' consent and violence in national legislatures''.}}

\begin{document}

\maketitle

%%%%%%% Abstract %%%%%%%%%%%%%%%%%%%%%%%%%%%%%%%%%%%%%%%%%%%%
\begin{abstract}
Multi-party elected national legislatures should be venues for peacefully resolving conflicts between opposing groups. However, they can become scenes of physical violence. Such violence is an indication that a country's legislative institutions are functioning far from perfectly as legislative actors are deciding to disregard the rules of the game. In some cases, such as recently in Ukraine, violence can indicate and possibly fuel deeper political divisions. In this first global study of legislative violence, I show that brawls are more likely when legislators find it difficult to credibly commit to follow peaceful bargains. Credible commitment problems are more acute in countries with disproportionate electoral outcomes and in new democracies. I find robust evidence for this argument using a case study of legislative violence in the antebellum United States Senate and a new global data set.
\end{abstract}


\paragraph{Keywords:} legislatures, violence, credible commitment problems, electoral proportionality, institutional design, majority and minority governments

\end{document}
